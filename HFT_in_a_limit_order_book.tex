\documentclass[10pt]{article}
\usepackage[english]{babel}
\usepackage[utf8x]{inputenc}
\usepackage[T1]{fontenc}
\usepackage{scribe}
\usepackage{listings}

\lstset{style=mystyle}
\setlength\parindent{0pt}
\begin{document}
\MakeScribeTop{Market MicroStructure (MM)}{High-frequency trading in a limit order book}{MARCO AVELLANEDA and SASHA STOIKOV*}
%#############################################################
%#############################################################
%#############################################################
%#############################################################



\section{Trading Order Book Dynamics} 

\subsection{Basics of Trading Order Book (TOB)}

\begin{itemize} 
    \item Buyers/Sellers express their intent to trade by submitting bids/asks
    \item These are Limit Orders (LO) with a price $P$ and size $N$
    \item Buy LO $(P, N)$ states willingness to buy $N$ shares at a price $\leq P$
    \item Sell LO $(P, N)$ states willingness to sell $N$ shares at a price $\geq P$
    \item Trading Order Book aggregates order sizes for each unique price
    \item So we can represent with two sorted lists of (Price, Size) pairs
    $$
    \begin{array}{l}
    \text { Bids: }\left[\left(P_{i}^{(b)}, N_{i}^{(b)}\right) \mid 1 \leq i \leq m\right], P_{i}^{(b)}>P_{j}^{(b)} \text { for } i<j \\
    \text { Asks: }\left[\left(P_{i}^{(a)}, N_{i}^{(a)}\right) \mid 1 \leq i \leq n\right], P_{i}^{(a)}<P_{j}^{(a)} \text { for } i<j
    \end{array}
    $$
    \item We call $P_{1}^{(b)}$ as simply Bid, $P_{1}^{(a)}$ as $A s k, \frac{P_{1}^{(a)}+P_{1}^{(b)}}{2}$ as $M i d$
    \item We call $P_{1}^{(a)}-P_{1}^{(b)}$ as Spread, $P_{n}^{(a)}-P_{m}^{(b)}$ as Market Depth
    \item A Market Order (MO) states intent to buy/sell $N$ shares at the best possible price(s) available on the $\mathrm{TOB}$ at the time of $\mathrm{MO}$ submission
\end{itemize}

\textbf{Trading Order Book (TOB) Activity}

\begin{itemize} 
    \item A new Sell LO $(P, N)$ potentially removes best bid prices on the TOB Removal: 
    $$
    \left[\left(P_{i}^{(b)}, \min \left(N_{i}^{(b)}, \max \left(0, N-\sum_{j=1}^{i-1} N_{j}^{(b)}\right)\right)\right) \mid\left(i: P_{i}^{(b)} \geq P\right)\right]
    $$
    \item After this removal, it adds the following to the asks side of the TOB
    $$
    \left(P, \max \left(0, N-\sum_{i: P_{i}^{(b)} \geq P} N_{i}^{(b)}\right)\right)
    $$
    \item A new Buy MO operates analogously (on the other side of the TOB)
    \item A Sell Market Order $N$ will remove the best bid prices on the TOB Removal: 
    $$
    \left[\left(P_{i}^{(b)}, \min \left(N_{i}^{(b)}, \max \left(0, N-\sum_{j=1}^{i-1} N_{j}^{(b)}\right)\right)\right) \mid 1 \leq i \leq m\right]
    $$
    \item A new Buy MO operates analogously (on the other side of the TOB)
    \item A Sell Market Order $N$ will remove the best bid prices on the TOB Removal: 
    $$
    \left[\left(P_{i}^{(b)}, \min \left(N_{i}^{(b)}, \max \left(0, N-\sum_{j=1}^{i-1} N_{j}^{(b)}\right)\right)\right) \mid 1 \leq i \leq m\right]
    $$
    \item A Buy Market Order $N$ will remove the best ask prices on the TOB
    $$
    \text { Removal: }\left[\left(P_{i}^{(a)}, \min \left(N_{i}^{(a)}, \max \left(0, N-\sum_{i=1}^{i-1} N_{j}^{(a)}\right)\right)\right) \mid 1 \leq i \leq n\right]
    $$
    
\end{itemize}

\section{Notation for Optimal Market-Making Problem}

\textbf{TOB Dynamics and Market-Making}

\begin{itemize} 
    \item Modeling TOB Dynamics involves predicting arrival of MOs and LOs
    \item Market-makers are liquidity providers (providers of Buy and Sell LOs)
    \item Other market participants are typically diquidity takers (MOs)
    \item But there are also other market participants that trade with LOs
    \item Complex interplay between market-makers \& other mkt participants
    \item Hence, TOB Dynamics tend to be quite complex
    \item We view the $\mathrm{TOB}$ from the perspective of a single market-maker who aims to gain with Buy/Sell LOs of appropriate width/size
    \item By anticipating TOB Dynamics \& dynamically adjusting Buy/Sell LOs
    \item Goal is to maximize Utility of Gains at the end of a suitable horizon
    \item If Buy/Sell LOs are too narrow, more frequent but small gains
    \item If Buy/Sell LOs are too wide, less frequent but large gains
    \item Market-maker also needs to manage potential unfavorable inventory (long or short) buildup and consequent unfavorable liquidation
\end{itemize}

\textbf{Notation for Optimal Market-Making Problem}

\begin{itemize} 
    \item We simplify the setting for ease of exposition
    \item Assume finite time steps indexed by $t=0,1, \ldots, T$
    \item Denote $W_{t} \in \mathbb{R}$ as Market-maker's trading $\mathrm{PnL}$ at time $t$
    \item Denote $I_{t} \in \mathbb{Z}$ as Market-maker's inventory of shares at time $t\left(I_{0}=0\right)$
    \item $S_{t} \in \mathbb{R}^{+}$ is the TOB Mid Price at time $t$ (assume stochastic process)
    \item $P_{t}^{(b)} \in \mathbb{R}^{+}, N_{t}^{(b)} \in \mathbb{Z}^{+}$ are market maker's Bid Price, Bid Size at time $t$
    \item $P_{t}^{(a)} \in \mathbb{R}^{+}, N_{t}^{(a)} \in \mathbb{Z}^{+}$ are market-maker's Ask Price, Ask Size at time $t$
    \item Assume market-maker can add or remove bids/asks costlessly
    \item Denote $\delta_{t}^{(b)}=S_{t}-P_{t}^{(b)}$ as Bid Spread, $\delta_{t}^{(a)}=P_{t}^{(a)}-S_{t}$ as Ask Spread
    \item Random $\operatorname{var} X_{t}^{(b)} \in \mathbb{Z}_{\geq 0}$ denotes bid-shares "hit" up to time $t$
    \item Random var $X_{t}^{(a)} \in \mathbb{Z}_{\geq 0}$ denotes ask-shares "lifted" up to time $t$ $W_{t+1}=W_{t}+P_{t}^{(a)} \cdot\left(X_{t+1}^{(a)}-X_{t}^{(a)}\right)-P_{t}^{(b)} \cdot\left(X_{t+1}^{(b)}-X_{t}^{(b)}\right), I_{t}=X_{t}^{(b)}-X_{t}^{(a)}$
    \item Goal to maximize $\mathbb{E}\left[U\left(W_{T}+I_{T} \cdot S_{T}\right)\right]$ for appropriate concave $U(\cdot)$
\end{itemize}

\section{Derivation of Avellaneda-Stoikov Analytical Solution}

\subsection{Avellaneda-Stoikov Continuous Time Formulation}

\begin{itemize} 
    \item Adapt our discrete-time notation to their continuous-time setting
    \item $X_{t}^{(b)}, X_{t}^{(a)}$ are Poisson processes with hit/lift-rate means $\lambda_{t}^{(b)}, \lambda_{t}^{(a)}$
    $$
    d X_{t}^{(b)} \sim \operatorname{Poisson}\left(\lambda_{t}^{(b)} \cdot d t\right), d X_{t}^{(a)} \sim \operatorname{Poisson}\left(\lambda_{t}^{(a)} \cdot d t\right)
    $$
    $\lambda_{t}^{(b)}=f^{(b)}\left(\delta_{t}^{(b)}\right), \lambda_{t}^{(a)}=f^{(a)}\left(\delta_{t}^{(a)}\right)$ for decreasing functions $f^{(b)}, f^{(a)}$
    $$
    d W_{t}=P_{t}^{(a)} \cdot d X_{t}^{(a)}-P_{t}^{(b)} \cdot d X_{t}^{(b)}, I_{t}=X_{t}^{(b)}-X_{t}^{(a)}\left(\text { note: } I_{0}=0\right)
    $$
    \item Since infinitesimal Poisson random variables $d X_{t}^{(b)}$ (shares hit in time $d t$ ) and $d X_{t}^{(a)}$ (shares lifted in time $d t$ ) are Bernoulli (shares hit/lifted in time $d t$ are 0 or 1$), N_{t}^{(b)}$ and $N_{t}^{(a)}$ can be assumed to be 1
    \item This simplifies the Action at time $t$ to be just the pair: $\left(\delta_{t}^{(b)}, \delta_{t}^{(a)}\right)$
    \item TOB Mid Price Dynamics: $d S_{t}=\sigma \cdot d z_{t}$ (scaled brownian motion)
    \item Utility function $U(x)=-e^{-\gamma x}$ where $\gamma>0$ is coeff. of risk-aversion
\end{itemize}


\subsection{Hamilton-Jacobi-Bellman (HJB) Equation}

\begin{itemize} 
    \item We denote the Optimal Value function as $V^{*}\left(t, S_{t}, W_{t}, l_{t}\right)$
    $$
    V^{*}\left(t, S_{t}, W_{t}, I_{t}\right)=\max _{\delta_{t}^{(b)}, \delta_{t}^{(a)}} \mathbb{E}\left[-e^{-\gamma \cdot\left(W_{T}+l_{t} \cdot S_{T}\right)}\right]
    $$
    \item $V^{*}\left(t, S_{t}, W_{t}, I_{t}\right)$ satisfies a recursive formulation for $0 \leq t<t_{1}<T$ :
    $$
    V^{*}\left(t, S_{t}, W_{t}, I_{t}\right)=\max _{\delta_{t}^{(b)}, \delta_{t}^{(a)}} \mathbb{E}\left[V^{*}\left(t_{1}, S_{t_{1}}, W_{t_{1}}, l_{t_{1}}\right)\right]
    $$
    \item Rewriting in stochastic differential form, we have the HJB Equation
    $$
    \begin{array}{c}
    \max _{\delta_{t}^{(b)}, \delta_{t}^{(a)}} \mathbb{E}\left[d V^{*}\left(t, S_{t}, W_{t}, I_{t}\right)\right]=0 \text { for } t<T \\
    V^{*}\left(T, S_{T}, W_{T}, I_{T}\right)=-e^{-\gamma \cdot\left(W_{T}+I_{T} \cdot S_{T}\right)}
    \end{array}
    $$
\end{itemize}

\subsection{Converting HJB to a Partial Dierential Equation}

\begin{itemize} 
    \item Change to $V^{*}\left(t, S_{t}, W_{t}, l_{t}\right)$ is comprised of 3 components:
    \begin{itemize} 
        \item Due to pure movement in time $t$
        \item Due to randomness in TOB Mid-Price $S_{t}$
        \item Due to randomness in hitting/lifting the Bid/Ask
    \end{itemize}
    \item With this, we can expand $d V^{*}\left(t, S_{t}, W_{t}, l_{t}\right)$ and rewrite HJB as:
    $$
    \begin{aligned}
    \max _{\delta_{t}^{(b)}, \delta_{t}^{(a)}}\{& \frac{\partial V^{*}}{\partial t} d t+\mathbb{E}\left[\sigma \frac{\partial V^{*}}{\partial S_{t}} d z_{t}+\frac{\sigma^{2}}{2} \frac{\partial^{2} V^{*}}{\partial S_{t}^{2}}\left(d z_{t}\right)^{2}\right] \\
    &+\lambda_{t}^{(b)} \cdot d t \cdot V^{*}\left(t, S_{t}, W_{t}-S_{t}+\delta_{t}^{(b)}, I_{t}+1\right) \\
    &+\lambda_{t}^{(a)} \cdot d t \cdot V^{*}\left(t, S_{t}, W_{t}+S_{t}+\delta_{t}^{(a)}, I_{t}-1\right) \\
    &+\left(1-\lambda_{t}^{(b)} \cdot d t-\lambda_{t}^{(a)} \cdot d t\right) \cdot V^{*}\left(t, S_{t}, W_{t}, l_{t}\right) \\
    &\left.-V^{*}\left(t, S_{t}, W_{t}, I_{t}\right)\right\}=0
    \end{aligned}
    $$
\end{itemize}

\begin{itemize} 
    \item We can simplify this equation with a few observations:
    \begin{itemize} 
        \item $\mathbb{E}\left[d z_{t}\right]=0$
        \item $\mathbb{E}\left[\left(d z_{t}\right)^{2}\right]=d t$
        \item Organize the terms involving $\lambda_{t}^{(b)}$ and $\lambda_{t}^{(a)}$ better with some algebra
        \item Divide throughout by $dt$
    \end{itemize}
    $$
    \begin{aligned} \max _{\delta_{t}^{(b)}, \delta_{t}^{(a)}}\{& \frac{\partial V^{*}}{\partial t}+\frac{\sigma^{2}}{2} \frac{\partial^{2} V^{*}}{\partial S_{t}^{2}} \\ &+\lambda_{t}^{(b)} \cdot\left(V^{*}\left(t, S_{t}, W_{t}-S_{t}+\delta_{t}^{(b)}, I_{t}+1\right)-V^{*}\left(t, S_{t}, W_{t}, I_{t}\right)\right) \\ & \left.+\lambda_{t}^{(a)} \cdot\left(V^{*}\left(t, S_{t}, W_{t}+S_{t}+\delta_{t}^{(a)}, I_{t}-1\right)-V^{*}\left(t, S_{t}, W_{t}, I_{t}\right)\right)\right\}=0 \end{aligned}
    $$
\end{itemize}

\begin{itemize} 
    \item Next, note that $\lambda_{t}^{(b)}=f^{(b)}\left(\delta_{t}^{(b)}\right)$ and $\lambda_{t}^{(a)}=f^{(a)}\left(\delta_{t}^{(a)}\right)$, and apply the max
    only on the relevant terms
    $$
    \frac{\partial V^{*}}{\partial t}+\frac{\sigma^{2}}{2} \frac{\partial^{2} V^{*}}{\partial S_{t}^{2}}
    $$
    $$
    +\max _{\delta_{t}^{(b)}}\left\{f^{(b)}\left(\delta_{t}^{(b)}\right) \cdot\left(V^{*}\left(t, S_{t}, W_{t}-S_{t}+\delta_{t}^{(b)}, l_{t}+1\right)-V^{*}\left(t, S_{t}, W_{t}, I_{t}\right)\right)\right\}\\
    $$
    $$
    +\max _{\delta_{t}^{(a)}}\left\{f^{(a)}\left(\delta_{t}^{(a)}\right) \cdot\left(V^{*}\left(t, S_{t}, W_{t}+S_{t}+\delta_{t}^{(a)}, I_{t}-1\right)-V^{*}\left(t, S_{t}, W_{t}, I_{t}\right)\right)\right\}=0\\
    $$
    \item This combines with the boundary condition:
    $$
    V^{*}\left(T, S_{T}, W_{T}, I_{T}\right)=-e^{-\gamma \cdot\left(W_{T}+I_{T} \cdot S_{T}\right)}
    $$
\end{itemize}

\begin{itemize} 
    \item We make an "educated guess" for the structure of $V^{*}\left(t, S_{t}, W_{t}, I_{t}\right)$
    \[
    V^{*}\left(t, S_{t}, W_{t}, I_{t}\right)=-e^{-\gamma\left(W_{t}+\theta\left(t, S_{t}, I_{t}\right)\right)} \tag{1}
    \]
    and reduce the problem to a PDE in terms of $\theta\left(t, S_{t}, l_{t}\right)$
    \item Substituting this into the above PDE for $V^{*}\left(t, S_{t}, W_{t}, I_{t}\right)$ gives:
    $$
    \begin{array}{l}
    \frac{\partial \theta}{\partial t}+\frac{\sigma^{2}}{2}\left(\frac{\partial^{2} \theta}{\partial S_{t}^{2}}-\gamma\left(\frac{\partial \theta}{\partial S_{t}}\right)^{2}\right) \\
    +\underset{\delta_{t}^{(b)}}{\max }\left\{\frac{f^{(b)}\left(\delta_{t}^{(b)}\right)}{\gamma} \cdot\left(1-e^{-\gamma\left(\delta_{t}^{(b)}-S_{t}+\theta\left(t, S_{t}, l_{t}+1\right)-\theta\left(t, S_{t}, I_{t}\right)\right)}\right)\right\} \\
    \left.+\max _{\left.\delta_{t}^{(a)}\right)} \frac{f^{(a)}\left(\delta_{t}^{(a)}\right)}{\gamma} \cdot\left(1-e^{-\gamma\left(\delta_{t}^{(a)}+S_{t}+\theta\left(t, S_{t}, I_{t}-1\right)-\theta\left(t, S_{t}, I_{t}\right)\right)}\right)\right\}=0
    \end{array}
    $$
    \item The boundary condition is:
    $$
    \theta\left(T, S_{T}, I_{T}\right)=I_{T} \cdot S_{T}
    $$
\end{itemize}

\subsection{Indifference Bid/Ask Price}

\begin{itemize} 
    \item It turns out that $\theta\left(t, S_{t}, I_{t}+1\right)-\theta\left(t, S_{t}, I_{t}\right)$ and $\theta\left(t, S_{t}, I_{t}\right)-\theta\left(t, S_{t}, l_{t}-1\right)$ are equal to financially meaningful quantities known as Indifference Bid and Ask Prices
    \item Indifference Bid Price $Q^{(b)}\left(t, S_{t}, I_{t}\right)$ is defined as:
    \[
    V^{*}\left(t, S_{t}, W_{t}-Q^{(b)}\left(t, S_{t}, I_{t}\right), I_{t}+1\right)=V^{*}\left(t, S_{t}, W_{t}, l_{t}\right) \tag{2}
    \]
    \item $Q^{(b)}\left(t, S_{t}, l_{t}\right)$ is the price to buy a share with guarantee of immediate purchase that results in Optimum Expected Utility being unchanged
    \item Likewise, Indifference Ask Price $Q^{(a)}\left(t, S_{t}, I_{t}\right)$ is defined as:
    \[
    V^{*}\left(t, S_{t}, W_{t}+Q^{(a)}\left(t, S_{t}, I_{t}\right), I_{t}-1\right)=V^{*}\left(t, S_{t}, W_{t}, I_{t}\right) \tag{3}
    \]
    \item $Q^{(a)}\left(t, S_{t}, l_{t}\right)$ is the price to sell a share with guarantee of immediate sale that results in Optimum Expected Utility being unchanged
    \item We abbreviate $Q^{(b)}\left(t, S_{t}, I_{t}\right)$ as $Q_{t}^{(b)}$ and $Q^{(a)}\left(t, S_{t}, I_{t}\right)$ as $Q_{t}^{(a)}$
\end{itemize}

\textbf{Indifference Bid/Ask Price in the PDE for $\theta$}

\begin{itemize} 
    \item Express $V^{*}\left(t, S_{t}, W_{t}-Q_{t}^{(b)}, l_{t}+1\right)=V^{*}\left(t, S_{t}, W_{t}, I_{t}\right)$ in terms of $\theta$ :
    \[
    \begin{array}{c}
    -e^{-\gamma\left(W_{t}-Q_{t}^{(b)}+\theta\left(t, S_{t}, l_{t}+1\right)\right)}=-e^{-\gamma\left(W_{t}+\theta\left(t, S_{t}, I_{t}\right)\right)} \\
    \Rightarrow Q_{t}^{(b)}=\theta\left(t, S_{t}, l_{t}+1\right)-\theta\left(t, S_{t}, l_{t}\right)
    \end{array} \tag{4}
    \]
    \item Likewise for $Q_{t}^{(a)}$, we get:
    \[
    Q_{t}^{(a)}=\theta\left(t, S_{t}, I_{t}\right)-\theta\left(t, S_{t}, I_{t}-1\right) \tag{5}
    \]
    \item Using equations (4) and (5), bring $Q_{t}^{(b)}$ and $Q_{t}^{(a)}$ in the PDE for $\theta$
    $$
    \begin{array}{c}
    \frac{\partial \theta}{\partial t}+\frac{\sigma^{2}}{2}\left(\frac{\partial^{2} \theta}{\partial S_{t}^{2}}-\gamma\left(\frac{\partial \theta}{\partial S_{t}}\right)^{2}\right)+\max _{\delta_{t}^{(b)}} g\left(\delta_{t}^{(b)}\right)+\max _{\delta_{t}^{(a)}} h\left(\delta_{t}^{(b)}\right)=0 \\
    \text { where } g\left(\delta_{t}^{(b)}\right)=\frac{f^{(b)}\left(\delta_{t}^{(b)}\right)}{\gamma} \cdot\left(1-e^{-\gamma\left(\delta_{t}^{(b)}-S_{t}+Q_{t}^{(b)}\right)}\right) \\
    \text { and } h\left(\delta_{t}^{(a)}\right)=\frac{f^{(a)}\left(\delta_{t}^{(a)}\right)}{\gamma} \cdot\left(1-e^{-\gamma\left(\delta_{t}^{(a)}+S_{t}-Q_{t}^{(a)}\right)}\right)
    \end{array}
    $$
\end{itemize}

\subsection{Optimal Bid Spread and Optimal Ask Spread}

\begin{itemize} 
    \item To maximize $g\left(\delta_{t}^{(b)}\right)$, differentiate $g$ with respect to $\delta_{t}^{(b)}$ and set to 0
    \[
    \begin{array}{c}
    e^{-\gamma\left(\delta_{t}^{(b)^{*}}-S_{t}+Q_{t}^{(b)}\right)} \cdot\left(\gamma \cdot f^{(b)}\left(\delta_{t}^{(b)^{*}}\right)-\frac{\partial f^{(b)}}{\partial \delta_{t}^{(b)}}\left(\delta_{t}^{(b)^{*}}\right)\right)+\frac{\partial f^{(b)}}{\partial \delta_{t}^{(b)}}\left(\delta_{t}^{(b)^{*}}\right)=0 \\
    \Rightarrow \delta_{t}^{(b)^{*}}=S_{t}-Q_{t}^{(b)}+\frac{1}{\gamma} \cdot \ln \left(1-\gamma \cdot \frac{f^{(b)}\left(\delta_{t}^{(b)^{*}}\right)}{\frac{\partial f^{(b)}}{\partial \delta_{t}^{(b)}}\left(\delta_{t}^{(b)^{*}}\right)}\right)
    \end{array} \tag{6}
    \]
    \item To maximize $g\left(\delta_{t}^{(a)}\right)$, differentiate $g$ with respect to $\delta_{t}^{(a)}$ and set to 0
    \[
    \begin{array}{c}
    e^{-\gamma\left(\delta_{t}^{(a)^{*}}+S_{t}-Q_{t}^{(a)}\right)} \cdot\left(\gamma \cdot f^{(a)}\left(\delta_{t}^{(a)^{*}}\right)-\frac{\partial f^{(a)}}{\partial \delta_{t}^{(a)}}\left(\delta_{t}^{(a)^{*}}\right)\right)+\frac{\partial f^{(a)}}{\partial \delta_{t}^{(a)}}\left(\delta_{t}^{(a)^{*}}\right)=0 \\
    \Rightarrow \delta_{t}^{(a)^{*}}=Q_{t}^{(a)}-S_{t}+\frac{1}{\gamma} \cdot \ln \left(1-\gamma \cdot \frac{f^{(a)}\left(\delta_{t}^{(a)^{*}}\right)}{\frac{\partial f^{(a)}}{\partial \delta_{t}^{(a)}}\left(\delta_{t}^{(a)^{*}}\right)}\right)
    \end{array} \tag{7}
    \]
    \item (6) and (7) are implicit equations for $\delta_{t}^{(b)^{*}}$ and $\delta_{t}^{(a)^{*}}$ respectively
\end{itemize}

\textbf{Solving for $\theta$ and for Optimal Bid/Ask Spreads}

\begin{itemize} 
    \item Let us write the PDE in terms of the Optimal Bid and Ask Spreads
    \[
    \begin{array}{l}
    \frac{\partial \theta}{\partial t}+\frac{\sigma^{2}}{2}\left(\frac{\partial^{2} \theta}{\partial S_{t}^{2}}-\gamma\left(\frac{\partial \theta}{\partial S_{t}}\right)^{2}\right) \\
    +\frac{f^{(b)}\left(\delta_{t}^{(b)^{*}}\right)}{\gamma} \cdot\left(1-e^{-\gamma\left(\delta_{t}^{(b)^{*}}-S_{t}+\theta\left(t, S_{t}, I_{t}+1\right)-\theta\left(t, S_{t}, l_{t}\right)\right)}\right) \\
    +\frac{f^{(a)}\left(\delta_{t}^{(a)^{*}}\right)}{\gamma} \cdot\left(1-e^{-\gamma\left(\delta_{t}^{(a)^{*}}+S_{t}+\theta\left(t, S_{t}, I_{t}-1\right)-\theta\left(t, S_{t}, I_{t}\right)\right)}\right)=0
    \end{array} \tag{8}
    \]
    \item with boundary condition $\theta\left(T, S_{T}, I_{T}\right)=I_{T} \cdot S_{T}$
    \item First we solve PDE (8) for $\theta$ in terms of $\delta_{t}^{(b)^{*}}$ and $\delta_{t}^{(a)^{*}}$
    \item In general, this would be a numerical PDE solution
    \item Using (4) and (5), we have $Q_{t}^{(b)}$ and $Q_{t}^{(a)}$ in terms of $\delta_{t}^{(b)^{*}}$ and $\delta_{t}^{(a)^{*}}$
    \item Substitute above-obtained $Q_{t}^{(b)}$ and $Q_{t}^{(a)}$ in equations (6) and (7)
    \item Solve implicit equations for $\delta_{t}^{(b)^{*}}$ and $\delta_{t}^{(a)^{*}}$ (in general, numerically)
\end{itemize}

\textbf{Building Intuition}

\begin{itemize} 
    \item Define Indifference Mid Price $Q_{t}^{(m)}=\frac{Q_{t}^{(b)}+Q_{t}^{(a)}}{2}$
    \item To develop intuition for Indifference Prices, consider a simple case where the market-maker doesn't supply any bids or asks
    $$
    V^{*}\left(t, S_{t}, W_{t}, l_{t}\right)=\mathbb{E}\left[-e^{-\gamma\left(W_{t}+l_{t} \cdot S_{T}\right)} \mid\right]
    $$
    \item Combining this with the diffusion $d S_{t}=\sigma \cdot d z_{t}$, we get:
    $$
    V^{*}\left(t, S_{t}, W_{t}, I_{t}\right)=-e^{-\gamma\left(W_{t}+l_{t} \cdot S_{t}-\frac{\gamma \cdot l_{t}^{2} \cdot \sigma^{2}(T-t)}{2}\right)}
    $$
    \item Combining this with equations $(2)$ and $(3)$, we get:
    $$
    \begin{array}{c}
    Q_{t}^{(b)}=S_{t}-\left(2 I_{t}+1\right) \frac{\gamma \sigma^{2}(T-t)}{2}, Q_{t}^{(a)}=S_{t}-\left(2 l_{t}-1\right) \frac{\gamma \sigma^{2}(T-t)}{2} \\
    Q_{t}^{(m)}=S_{t}-I_{t} \gamma \sigma^{2}(T-t), Q_{t}^{(a)}-Q_{t}^{(b)}=\gamma \sigma^{2}(T-t)
    \end{array}
    $$
    \item These results for the simple case of no-market-making serve as approximations for our problem of optimal market-making
\end{itemize}

\begin{itemize} 
    \item Think of $Q_{t}^{(m)}$ as inventory-risk-adjusted mid-price (adjustment to $\left.S_{t}\right)$
    \item If market-maker is long inventory $\left(I_{t}>0\right), Q_{t}^{(m)}<S_{t}$ indicating inclination to sell than buy, and if market-maker is short inventory, $Q_{t}^{(m)}>S_{t}$ indicating inclination to buy than sell
    \item Armed with this intuition, we come back to optimal market-making, observing from eqns $(6)$ and $(7): P_{t}^{(b)^{*}}<Q_{t}^{(b)}<Q_{t}^{(m)}<Q_{t}^{(a)}<P_{t}^{(a)^{*}}$
    \item Think of $\left[P_{t}^{(b)^{*}}, P_{t}^{(a)^{*}}\right]$ as "centered" at $Q_{t}^{(m)}$ (rather than at $\left.S_{t}\right)$,
    i.e., $\left[P_{t}^{(b)^{*}}, P_{t}^{(a)^{*}}\right]$ will (together) move up/down in tandem with $Q_{t}^{(m)}$ moving up/down (as a function of inventory position $\left.I_{t}\right)$
    \[
    \begin{array}{c}
    Q_{t}^{(m)}-P_{t}^{(b)^{*}}=\frac{Q_{t}^{(a)}-Q_{t}^{(b)}}{2}+\frac{1}{\gamma} \cdot \ln \left(1-\gamma \cdot \frac{f^{(b)}\left(\delta_{t}^{(b)^{*}}\right)}{\frac{\partial f^{(b)}}{\partial \delta_{t}^{(b)}}\left(\delta_{t}^{(b)^{*}}\right)}\right) \tag{9,10}\\
    P_{t}^{(a)^{*}}-Q_{t}^{(m)}=\frac{Q_{t}^{(a)}-Q_{t}^{(b)}}{2}+\frac{1}{\gamma} \cdot \ln \left(1-\gamma \cdot \frac{f^{(a)}\left(\delta_{t}^{(a)^{*}}\right)}{\frac{\partial f(a)}{\partial \delta_{t}^{(a)}}\left(\delta_{t}^{(a)^{*}}\right)}\right)
    \end{array} 
    \]
\end{itemize}

\subsection{Simple Functional Form for Hitting/Lifting Rate Means}

\begin{itemize} 
    \item The PDE for $\theta$ and the implicit equations for $\delta_{t}^{(b)^{*}}, \delta_{t}^{(a)^{*}}$ are messy
    \item We make some assumptions, simplify, derive analytical approximations
    \item First we assume a fairly standard functional form for $f^{(b)}$ and $f^{(a)}$
    $$
    f^{(b)}(\delta)=f^{(a)}(\delta)=c \cdot e^{-k \cdot \delta}
    $$
    \item This reduces equations $(6)$ and $(7)$ to:
    \[
    \begin{array}{l}
    \delta_{t}^{(b)^{*}}=S_{t}-Q_{t}^{(b)}+\frac{1}{\gamma} \ln \left(1+\frac{\gamma}{k}\right) \tag{11,12} \\
    \delta_{t}^{(a)^{*}}=Q_{t}^{(a)}-S_{t}+\frac{1}{\gamma} \ln \left(1+\frac{\gamma}{k}\right)
    \end{array}
    \]
    $$
    \Rightarrow P_{t}^{(b)^{*}} and P_{t}^{(a)^{*}} are\, equidistant\, from\, Q_{t}^{(m)}
    $$
    \item Substituting these simplified $\delta_{t}^{(b)^{*}}, \delta_{t}^{(a)^{*}}$ in (8) reduces the PDE to:
    \[
    \frac{\partial \theta}{\partial t}+\frac{\sigma^{2}}{2}\left(\frac{\partial^{2} \theta}{\partial S_{t}^{2}}-\gamma\left(\frac{\partial \theta}{\partial S_{t}}\right)^{2}\right)+\frac{c}{k+\gamma}\left(e^{-k \cdot \delta_{t}^{(b)^{*}}}+e^{-k \cdot \delta_{t}^{(a)^{*}}}\right)=0 \tag{13}
    \]
    \item with boundary condition $\theta\left(T, S_{T}, I_{T}\right)=I_{T} \cdot S_{T}$
\end{itemize}

\subsection{Simplifying the PDE with Approximations}

\begin{itemize} 
    \item Note that this PDE (13) involves $\delta_{t}^{(b)^{*}}$ and $\delta_{t}^{(a)^{*}}$
    \item However, equations $(11),(12),(4),(5)$ enable expressing $\delta_{t}^{(b)^{*}}$ and $\delta_{t}^{(a)^{*}}$ in terms of $\theta\left(t, S_{t}, I_{t}-1\right), \theta\left(t, S_{t}, I_{t}\right), \theta\left(t, S_{t}, I_{t}+1\right)$
    \item This would give us a PDE just in terms of $\theta$
    \item Solving that PDE for $\theta$ would not only give us $V^{*}\left(t, S_{t}, W_{t}, l_{t}\right)$ but also $\delta_{t}^{(b)^{*}}$ and $\delta_{t}^{(a)^{*}}$ (using equations $\left.(11),(12),(4),(5)\right)$
    \item To solve the PDE, we need to make a couple of approximations
    \item First we make a linear approx for $e^{-k \cdot \delta_{t}^{(b)^{*}}}$ and $e^{-k \cdot \delta_{t}^{(a)^{*}}}$ in PDE (13):
    \[
    \frac{\partial \theta}{\partial t}+\frac{\sigma^{2}}{2}\left(\frac{\partial^{2} \theta}{\partial S_{t}^{2}}-\gamma\left(\frac{\partial \theta}{\partial S_{t}}\right)^{2}\right)+\frac{c}{k+\gamma}\left(1-k \cdot \delta_{t}^{(b)^{*}}+1-k \cdot \delta_{t}^{(a)^{*}}\right)=0 \tag{14}
    \]
    \item Equations $(11),(12),(4),(5)$ tell us that:
    $$
    \delta_{t}^{(b)^{*}}+\delta_{t}^{(a)^{*}}=\frac{2}{\gamma} \ln \left(1+\frac{\gamma}{k}\right)+2 \theta\left(t, S_{t}, I_{t}\right)-\theta\left(t, S_{t}, I_{t}+1\right)-\theta\left(t, S_{t}, I_{t}-1\right)
    $$
\end{itemize}



\subsection{Asymptotic Expansion of $\theta$ in $I_{t}$}

\begin{itemize} 
    \item With this expression for $\delta_{t}^{(b)^{*}}+\delta_{t}^{(a)^{*}}$, PDE (14) takes the form:
    \[
    \begin{aligned}
    \frac{\partial \theta}{\partial t}+\frac{\sigma^{2}}{2} &\left(\frac{\partial^{2} \theta}{\partial S_{t}^{2}}-\gamma\left(\frac{\partial \theta}{\partial S_{t}}\right)^{2}\right)+\frac{c}{k+\gamma}\left(2-\frac{2 k}{\gamma} \ln \left(1+\frac{\gamma}{k}\right)\right.\\
    &\left.-k\left(2 \theta\left(t, S_{t}, I_{t}\right)-\theta\left(t, S_{t}, l_{t}+1\right)-\theta\left(t, S_{t}, I_{t}-1\right)\right)\right)=0
    \end{aligned} \tag{15}
    \]
    \item To solve PDE (15), we consider this asymptotic expansion of $\theta$ in $I_{t}$ :
    $$
    \theta\left(t, S_{t}, I_{t}\right)=\sum_{n=0}^{\infty} \frac{l_{t}^{n}}{n !} \cdot \theta^{(n)}\left(t, S_{t}\right)
    $$
    \item So we need to determine the functions $\theta^{(n)}\left(t, S_{t}\right)$ for all $n=0,1,2, \ldots$
    \item For tractability, we approximate this expansion to the first 3 terms:
    $$
    \theta\left(t, S_{t}, I_{t}\right) \approx \theta^{(0)}\left(t, S_{t}\right)+I_{t} \cdot \theta^{(1)}\left(t, S_{t}\right)+\frac{I_{t}^{2}}{2} \cdot \theta^{(2)}\left(t, S_{t}\right)
    $$
\end{itemize}

\subsection{Approximation of $\theta$ in $I_{t}$}

\begin{itemize} 
    \item We note that the Optimal Value Function $V^{*}$ can depend on $S_{t}$ only through the current Value of the Inventory (i.e., through $\left.I_{t} \cdot S_{t}\right)$, i.e., it cannot depend on $S_{t}$ in any other way
    \item This means $V^{*}\left(t, S_{t}, W_{t}, 0\right)=-e^{-\gamma\left(W_{t}+\theta^{(0)}\left(t, S_{t}\right)\right)}$ is independent of $S_{t}$
    \item This means $\theta^{(0)}\left(t, S_{t}\right)$ is independent of $S_{t}$
    \item So, we can write it as simply $\theta^{(0)}(t)$, meaning $\frac{\partial \theta^{(0)}}{\partial S_{t}}$ and $\frac{\partial^{2} \theta^{(0)}}{\partial S_{t}^{2}}$ are 0
    \item Therefore, we can write the approximate expansion for $\theta\left(t, S_{t}, I_{t}\right)$ as:
    \[
    \theta\left(t, S_{t}, I_{t}\right)=\theta^{(0)}(t)+I_{t} \cdot \theta^{(1)}\left(t, S_{t}\right)+\frac{I_{t}^{2}}{2} \cdot \theta^{(2)}\left(t, S_{t}\right) \tag{16}
    \]
\end{itemize}

\subsection{Solving the PDE}

\begin{itemize} 
    \item Substitute this approximation (16) for $\theta\left(t, S_{t}, I_{t}\right)$ in PDE (15)
    $$
    \begin{array}{l}
    \frac{\partial \theta^{(0)}}{\partial t}+I_{t} \frac{\partial \theta^{(1)}}{\partial t}+\frac{l_{t}^{2}}{2} \frac{\partial \theta^{(2)}}{\partial t}+\frac{\sigma^{2}}{2}\left(I_{t} \frac{\partial^{2} \theta^{(1)}}{\partial S_{t}^{2}}+\frac{l_{t}^{2}}{2} \frac{\partial^{2} \theta^{(2)}}{\partial S_{t}^{2}}\right) \\
    -\frac{\gamma \sigma^{2}}{2}\left(I_{t} \frac{\partial \theta^{(1)}}{\partial S_{t}}+\frac{l_{t}^{2}}{2} \frac{\partial \theta^{(2)}}{\partial S_{t}}\right)^{2}+\frac{c}{k+\gamma}\left(2-\frac{2 k}{\gamma} \ln \left(1+\frac{\gamma}{k}\right)+k \cdot \theta^{(2)}\right)=0
    \end{array}
    $$
    \item with boundary condition:
    \[
    \theta^{(0)}(T)+I_{T} \cdot \theta^{(1)}\left(T, S_{T}\right)+\frac{I_{T}^{2}}{2} \cdot \theta^{(2)}\left(T, S_{T}\right)=I_{T} \cdot S_{T} \tag{17}
    \]
    \item We will separately collect terms involving specific powers of $I_{t}$, each yielding a separate PDE:
    \begin{itemize} 
        \item Terms devoid of $I_{t}$ (i.e., $\left.I_{t}^{0}\right)$
        \item Terms involving $I_{t}$ (i.e., $I_{t}^{1}$ )
        \item Terms involving $I_{t}^{2}$
    \end{itemize}
\end{itemize}

\begin{itemize} 
    \item We start by collecting terms involving $I_{t}$
    $$
    \frac{\partial \theta^{(1)}}{\partial t}+\frac{\sigma^{2}}{2} \cdot \frac{\partial^{2} \theta^{(1)}}{\partial S_{t}^{2}}=0 with\, boundary\, condition\, \theta^{(1)}\left(T, S_{T}\right)=S_{T}
    $$
    \item The solution to this PDE is:
    \[
    \theta^{(1)}\left(t, S_{t}\right)=S_{t} \tag{18}
    \]
    \item Next, we collect terms involving $l_{t}^{2}$
    $$
    \frac{\partial \theta^{(2)}}{\partial t}+\frac{\sigma^{2}}{2} \cdot \frac{\partial^{2} \theta^{(2)}}{\partial S_{t}^{2}}-\gamma \sigma^{2} \cdot\left(\frac{\partial \theta^{(1)}}{\partial S_{t}}\right)^{2}=0 with\, boundary\, \theta^{(2)}\left(T, S_{T}\right)=0
    $$
    \item Noting that $\theta^{(1)}\left(t, S_{t}\right)=S_{t}$, we solve this $\mathrm{PDE}$ as:
    \[
    \theta^{(2)}\left(t, S_{t}\right)=-\gamma \sigma^{2}(T-t) \tag{19}
    \]
\end{itemize}

\begin{itemize} 
    \item Finally, we collect the terms devoid of $I_{t}$
    $$
    \frac{\partial \theta^{(0)}}{\partial t}+\frac{c}{k+\gamma}\left(2-\frac{2 k}{\gamma} \ln \left(1+\frac{\gamma}{k}\right)+k \cdot \theta^{(2)}\right)=0 \text { with boundary } \theta^{(0)}(T)=0
    $$
    \item Noting that $\theta^{(2)}\left(t, S_{t}\right)=-\gamma \sigma^{2}(T-t)$, we solve as:
    \[
    \theta^{(0)}(t)=\frac{c}{k+\gamma}\left(\left(2-\frac{2 k}{\gamma} \ln \left(1+\frac{\gamma}{k}\right)\right)(T-t)-\frac{k \gamma \sigma^{2}}{2}(T-t)^{2}\right) \tag{20}
    \]
    \item This completes the PDE solution for $\theta\left(t, S_{t}, l_{t}\right)$ and hence, for $V^{*}\left(t, S_{t}, W_{t}, l_{t}\right)$
    \item Lastly, we derive formulas for $Q_{t}^{(b)}, Q_{t}^{(a)}, Q_{t}^{(m)}, \delta_{t}^{(b)^{*}}, \delta_{t}^{(a)^{*}}$
\end{itemize}

\subsection{Formulas for Prices and Spreads}

\begin{itemize} 
    \item Using equations (4) and (5), we get:
    \[
    \begin{array}{l}
    Q_{t}^{(b)}=\theta^{(1)}\left(t, S_{t}\right)+\left(2 I_{t}+1\right) \cdot \theta^{(2)}\left(t, S_{t}\right)=S_{t}-\left(2 I_{t}+1\right) \frac{\gamma \sigma^{2}(T-t)}{2} \tag{21,22} \\
    Q_{t}^{(a)}=\theta^{(1)}\left(t, S_{t}\right)+\left(2 I_{t}-1\right) \cdot \theta^{(2)}\left(t, S_{t}\right)=S_{t}-\left(2 I_{t}-1\right) \frac{\gamma \sigma^{2}(T-t)}{2}
    \end{array}
    \]
    \item Using equations $(11)$ and $(12)$, we get:
    \[
    \delta_{t}^{(b)^{*}}=\frac{\left(2 I_{t}+1\right) \gamma \sigma^{2}(T-t)}{2}+\frac{1}{\gamma} \ln \left(1+\frac{\gamma}{k}\right) \tag{23}
    \]
    \[
    \delta_{t}^{(a)^{*}}=\frac{\left(1-2 I_{t}\right) \gamma \sigma^{2}(T-t)}{2}+\frac{1}{\gamma} \ln \left(1+\frac{\gamma}{k}\right) \tag{24}
    \]
    \item Optimal Bid-Ask Spread \[\delta_{t}^{(b)^{*}}+\delta_{t}^{(a)^{*}}=\gamma \sigma^{2}(T-t)+\frac{2}{\gamma} \ln \left(1+\frac{\gamma}{k}\right) \tag{25} \]
    \[
    \text { Optimal "Mid" } Q_{t}^{(m)}=\frac{Q_{t}^{(b)}+Q_{t}^{(a)}}{2}=\frac{P_{t}^{(b)^{*}}+P_{t}^{(a)^{*}}}{2}=S_{t}-I_{t} \gamma \sigma^{2}(T-t) \tag{26}
    \]
\end{itemize}

\section{Summary}

\begin{itemize} 
    \item Think of $Q_{t}^{(m)}$ as inventory-risk-adjusted mid-price (adjustment to $\left.S_{t}\right)$
    \item If market-maker is long inventory $\left(I_{t}>0\right), Q_{t}^{(m)}<S_{t}$ indicating inclination to sell than buy, and if market-maker is short inventory, $Q_{t}^{(m)}>S_{t}$ indicating inclination to buy than sell
    \item Think of $\left[P_{t}^{(b)^{*}}, P_{t}^{(a)^{*}}\right]$ as "centered" at $Q_{t}^{(m)}$ (rather than at $\left.S_{t}\right)$,
    i.e., $\left[P_{t}^{(b)^{*}}, P_{t}^{(a)^{*}}\right]$ will (together) move up/down in tandem with $Q_{t}^{(m)}$ moving up/down (as a function of inventory position $\left.I_{t}\right)$
    \item Note from equation (25) that the Optimal Bid-Ask Spread $P_{t}^{(a)^{*}}-P_{t}^{(b)^{*}}$ is independent of inventory $I_{t}$
    \item Useful view: $P_{t}^{(b)^{*}}<Q_{t}^{(b)}<Q_{t}^{(m)}<Q_{t}^{(a)}<P_{t}^{(a)^{*}}$, with these spreads:
    \begin{itemize} 
        \item Outer Spreads $P_{t}^{(a)^{*}}-Q_{t}^{(a)}=Q_{t}^{(b)}-P_{t}^{(b)^{*}}=\frac{1}{\gamma} \ln \left(1+\frac{\gamma}{k}\right)$
        \item Inner Spreads $Q_{t}^{(a)}-Q_{t}^{(m)}=Q_{t}^{(m)}-Q_{t}^{(b)}=\frac{\gamma \sigma^{2}(T-t)}{2}$
    \end{itemize}
\end{itemize}


\bibliographystyle{abbrv}           
\bibliography{./myref.bib}
\end{document}